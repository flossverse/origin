This chapter identifies an intersectional space across the described technologies, and proposes a valuable and novel software stack, which can enable exploration  and product development. It is useful to briefly look at some of the potential applications which might benefit from value and trust exchange within an global shared social space.

\begin{figure*}[ht]\centering 	\includegraphics[width=\linewidth]{highlevelstack}
	\caption{High level overview showing the components for sats, stablecoins on lightning, asssets, and trust}
	\label{fig:highlevelstack}
\end{figure*}

\section{Security evaluation}
As part of developing our stack we will penetration test the deployment as detailed using \href{https://hexway.io/}{Hexway}
\section{Our socialisation best practice}
\subsubsection{Identity}
\subsubsection{Webs of trust}
\subsubsection{Integration of 'good' actor AI entities}
Gratitude practice
Tipping and trust nudging
Feedback into web of trust
\subsection{Emulation of important social cues}
\subsubsection{Behaviour incentives, arbitration, and penalties}

\section{Potential applications }
\begin{itemize}
\item Art / NFT galleries with instant sales
\end{itemize}

This application allows artists and content creator communities to
display and sell NFT and fungible art to global consumer audiences,
instantly.

\begin{itemize}
\item
  Large scale conference center

  \begin{itemize}
  \item
    Academic conferences
  \item
    Political conference
  \item
    Commercial expo
  \end{itemize}
\end{itemize}

In a hypothetical virtual conference centre a true marketplace of ideas
could be enacted, with participants being paid directly by their
audience based on the proximity to the presentation.

\begin{itemize}
\item
  Group entertainment

  \begin{itemize}
  \item Global social puzzle gaming with prizes
  \item
    Music festivals and gigs - Pay live artists and DJs in real time
    depending on location within the extended landscape of the venue.
    Split to music producer a portion of the value
  \item
    Mixed reality theatre
  \item
    murder mystery
  \item
    Mixed reality live immersive MMORG games
  \item
    Bingo and mass participation gameshows
  \item
    Immersive brand storytelling metaverses
  \item
    Escape rooms
  \end{itemize}
\item
  Debating townhall meetings (with voting etc)
\item
  Mixed reality information metaverse

  \begin{itemize}
  \item
    AR based city tours with collectibles
  \item
    AR based collectibles for trails and heritage (museums, libraries)
    with location specific donations.
  \end{itemize}
\item
  Retail applications

  \begin{itemize}
  \item
    Proxy for physical market
  \item
    AR home delivery market interface within physical marketplaces
  \end{itemize}
\item
  Global course / Education provision
    \begin{itemize}
  \item
    Explore the universe as a group of spaceship or planet characters
  \item
    Explore biology and physics at a microscopic and nanoscopic level
  \end{itemize}
\item
  Micro tasking marketplace
\item
  Code bounty marketplace
\item
  Micro remittance role sharing (business PA / reception etc)
\item
  Careers fair with credential passing
\item
  Auctions in mixed reality
\item
  eSports and live sports
\item
  Gambling, betting markets, and financial leverage markets
\end{itemize}

\subsection{Global cybersec course delivery}
Isolating and building out one example here:
\begin{itemize}
\item Elements for the infrastructure: Economic layer, asset layer, content interface, user management, data storage, microsites loaded in Wolvin and webm, accessibility schema, network security, backups, secure messaging. Deployable framework with high modularity. Some more ossified elements for surity, some less so for malleability and open opportunity. Figure \ref{fig:globalclassroom}.
\item Course delivery in XR, how to we develop a platform, marketplace, framework for open contribution.
\item WebXR, Vircadia, any snap in metaverse middleware that is free and open source (action to compare the two). 
\item Define an interface schema for bolting in any commercial or FOSS metaverse engine.
\item VR marketplace (outside the scope of the VR engine) without a trusted third party.
\item Cryptographically managed learning deliverables (coursework as NFT). 
\item Secure messaging and group messaging using cryptographic keys. Check this stuff with the distributed computing science people in the group (action on John)
\item work toward an exemplar MVP which is then "in the wild"
\item Platform for educators
\item Define scheme, documentation, best practice, interfaces, functional objects, pedagogy, accessibility, multi-language. 
\item Define user management system for educators and client learners.
\item Identify the pain points which current FOSS elements which need development time/money
\item separate the UI/engine from the graphical assets, and the educational / pedagogical components, accessibility, and the value and asset transfer layers.
\item Desktop systems are the primary target (low end system)
\item define schema for accessibility. Colour, subtitles, immersion concerns which can be applied to metaverse rooms through API?
\item Start to define the hybrid presentation model we favour. Avatars? Micro sites? A combination of the two? Balance of guided vs unguided experience. Do we need to test the correct way to do delivery? Is there prior art we can draw on? I feel I should know. Is this part of the research that's being done here?
\item Big work package on schema vs key and user management to enforce rules in spaces. Only participants who have provably paid should have access to learning material, the ability to input into the assessment system, and the tokenised learning outcome `NFT' or proof.
\item Proof that XR system improve learning outcomes. Also that the proposed systems for micro-transactions and user and schema management give additional headroom for teaching.
\end{itemize}

Notes on build-out
The world database in the shared rooms in the metaverse is the global object master,  educational materials, videos,  audio content and branded objects are fungible tokens authentically proved by rgb client side validation between parties,  only validated ones will be persisted in shared rooms like conferences and classes according to the room schema. That allows educators to monetise their content.  That can work on lightning.  NFT objects between parties like content crafted by participants (coursework, homework) are not on lightning and will attract main chain fees but are rarer. User authentication and communication will be through nostr.

\begin{figure*}[ht]\centering % Using \begin{figure*} makes the figure take up the entire width of the page
	\includegraphics[width=\linewidth]{globalclassroom}
	\caption{Functional elements for infrastructure.}
	\label{fig:globalclassroom}
\end{figure*}

\begin{figure*}[ht]\centering % Using \begin{figure*} makes the figure take up the entire width of the page
	\includegraphics[width=\linewidth]{systemc4}
	\caption{Client server C4 diagrams.}
	\label{fig:globalclassroom}
\end{figure*}
