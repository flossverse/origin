\section{Summary TL;DR}
\label{sec:tldr}
\textbf{The points below are the tl;dr for the whole book. They do not currently correspond to the order in the book, but will do at some point. Work in progress.
\begin{itemize}
\item There may be an inflection point in the organisational topology of the internet because of trust abuses by the incumbent providers. This is calling itself Web3, but the moniker is fraught with problems and somewhat meaningless. The drivers are real.
\item `The Metaverse' is coming, in some form, at some point. Everyone is positioning in case it's ``soon''.
\item It's not at all clear what it is, or if people want it, but the best emergent narrative seems to be ``digital society'' and that obviously should not be dismissed lightly.
\item From a business perspective metaverse is the worst of the remote collaboration tool-kits, and undermines flow, productivity, and interpersonal trust.
\item Metaverse is probably technology for technologies sake at this time, but it's a strong attempt, and the investment is real. Digital society is more tangible but less hyped.
\item Excluding Facebook/Meta a lot of the investment is coming from the recent Web3 speculative bubble, who has a parallel and intersectional metaverse narrative. 
\item Industry has noted the risk, and failures of Meta in this domain, and have latched onto "open metaverse" as a narrative, to de-risk their interest.
\item The current open metaverse is muddy and confused.
\item Anything from a multi-million pound XR studio screen, to a speech audio system, can be a metaverse interface. AI is further blurring these boundaries.
%\item Constraints (metaverse schema) which are crafted in response to the use case, the interface, and the audience, is where challenges are most profound. This is an opportunity.
\item Trust, accessibility, governance, and safeguarding, are hard problems, and made more complex by unrecorded social flow in immersive social VR.
\item Industry seem to be missing the point; that open metaverse should mean open source metaverse. There are some options, but they are under developed. This is an opportunity.
\item There's a bunch of start-ups and options, interesting research, and some adoption, but scaling is very hard in the face of tepid consumer interest. We need to be driven by use cases.
\item There is genuine, undeniable interest in digital scarcity. The ownership of digital goods seems natural to younger, digitally native users. This is serviced already by various (gaming) platforms, but they are all isolated ecosystems.
\item Uniting these attempts, with portable (transferable) ``goods'' across digital society likely requires a global digital ledger (blockchain), indeed this is the basis of the competing Web3 interpretation. Crypto is igniting imagination on this topic, and is seeing adoption both inside out outside of the metaverse contexts.
\item Crypto is a nightmare; rife with scams, poor technology choices, limited life, and incorrect assumptions. The only thing blockchain/crypto can do well is ``money like networks'', which is a cornerstone of human interaction, and the killer application.
\item Representations of dollars and pounds etc money can ride securely on top of such networks as stablecoins, and this is getting easier to integrate, though there are risks.
\item It's unclear which technology will win, if any, but since the tools exist now they can be integrated now.
\item Legislative and cultural headwinds are significant. There might be no opportunity here in the end, though ``rough game theory'' supports the attempt.
\item Money, identity, and thereby trust, can already be mediated by the Bitcoin network, even without using Bitcoin the asset. This is an opportunity.
\item Digital ownership is nearly possible at scale, low cost, and minimal environmental impact, with the Bitcoin network. This is an opportunity.
\item A topologically flat, inclusive, permissionless, open metaverse, with economically empowered ML and AI actors, which can mediate governance issues, transparently, according to well constructed custom schemas, between cryptographically verifiable economic users (human or AI) is an opportunity.
\item New open source machine learning tooling potentially removes many of the problems with accessibility, creativity, language barriers, safeguarding, and governance. This is a huge opportunity.
\item Using new image generation ML it may be possible to build a new kind of collaborative global teams for virtual production, ideating in simplistic immersive spaces while instantly creating demonstrable in camera scenes which can be stylised using verbal commands in real-time.
\item Such teams would could be far more ad-hoc by leveraging the designs outlines in this book. 
\end{itemize}
}
