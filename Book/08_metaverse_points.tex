\section{TL;DR}
\textbf{The points below are the tl;dr for the whole book. They do not currently correspond to the order in the book, but will do at some point
\begin{itemize}
\item There may be an inflection point in the organisational topology of the internet because of trust abuses by the incumbent providers. This is calling itself Web3, but the moniker is fraught with problems and somewhat meaningless. The drivers are real.
\item `The Metaverse' is coming, in some form, at some point. Everyone is positioning in case it's ``soon''.
\item It's not at all clear what it is, or if people want it.
\item From a business perspective metaverse is the worst of the remote collaboration tool-kits, and undermines flow, productivity, and interpersonal trust.
\item It's probably technology for technologies sake to a degree, but the time feels right for the attempt, and the investment is real.
\item Industry has noted the risk, and failures of Meta (ex Facebook) in this domain, and have latched onto "open metaverse" as a narrative, to de-risk their interest.
\item Anything from a multi-million pound XR studio screen, to a speech audio system, can be a metaverse interface.
\item Constraints (metaverse schema) which are crafted in response to the use case, the interface, and the audience, is where challenges are most profound. This is an opportunity.
\item Trust, governance, and safeguarding, are far harder in these real-time distributed social systems.
\item There's a bunch of start-ups and options, interesting research, and some adoption, but scaling is very hard in the face of tepid consumer interest.
\item Industry seem to be missing the point; that open metaverse should mean open source metaverse. There are some options, but they are under developed. This is an opportunity.
\item There is genuine, undeniable interest in digital scarcity. The ownership of digital goods seems natural to younger, digitally native users.
\item This is serviced already by various (gaming) platforms, but they are all isolated ecosystems.
\item Uniting these attempts, with portable (transferable) ``goods'' across the digital universe (metaverse) likely requires a global digital ledger (blockchain).
\item Crypto is igniting imagination on this topic, and is seeing adoption both inside out outside of the metaverse contexts.
\item Crypto is a nightmare; rife with scams, poor technology choices, limited life, and incorrect assumptions.
\item The only thing blockchain/crypto can do well is ``money like networks'', which is a cornerstone of human interaction, and the killer application.
\item There do seem to be better options emerging. It's unclear which technology will win, if any.
\item Legislative and cultural headwinds are significant. There might be no opportunity here in the end, though ``rough game theory'' supports the attempt.
\item Money, identity, and thereby trust, can already be mediated by the Bitcoin network, even without using Bitcoin the asset. This is an opportunity.
\item Digital ownership is nearly possible at scale, low cost, and minimal environmental impact, with the Bitcoin network. This is an opportunity.
\item A topologically flat, inclusive, permissionless, open metaverse, with economically empowered ML and AI actors, which can mediate governance issues, transparently, according to well constructed custom schemas, between cryptographically verifiable economic users (human or AI) is an opportunity.
\end{itemize}
}
