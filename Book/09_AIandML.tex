\section{Augmented intelligence and ML}
\subsection{Novel VP render pipeline}
Putting the ML image generation on the end of a real-time tracked camera render pipeline might remove the need for detail in set building. To describe how this might work, the set designer, DP, director, etc will be able to ideate in a headset based metaverse of the set design, dropping very basic chairs, windows, light sources whatever. There is -no need- then to create a scene in detail. If the interframe consistency (img2img) can deliver then the output on the VP screen can simply inherit the artistic style from the text prompts, and render production quality from the basic building blocks. Everyone in the set (or just DP/director) could then switch in headset to the final output and ideate (verbally) to create the look and feel (lens, bokeh, light, artistic style etc). This isn’t ready yet as the frames need to generate much faster (100x), but it’s very likely coming in months not years. This ``next level pre-vis'' is being trailed by Pathway in the Vircadia collaborative environment described in this book, and can be seen illustrated in Figure \ref{fig:vircadiasd}.\par
\begin{figure}[ht]\centering 	\includegraphics[width=\linewidth]{vircadiasd}
	\caption{Top panel is a screen grab from Vircadia and the bottom panel is a quick pass through img2img from Stable Diffusion.}
	\label{fig:vircadiasd}
\end{figure}

This can be done now through the use of camera robots. A scene can be built in basic outline, the camera tracks can be encoded into the robot, and the scene can be rapidly post rendered by Stability with high inter frame consistency.\par
With the help of AI projects such as \href{https://nv-tlabs.github.io/LION/}{LION} it may be possible to pass simple geometry and instructions to ML systems which can create complex textured geometry back into the scene.
\begin{figure}[ht]\centering 	\includegraphics[width=\linewidth]{robotvp}
	\caption{Robot VP}
	\label{fig:robotvp}
\end{figure}
\subsection{Accessibility}
\subsubsection{Real time transcription}
\subsubsection{Real time translation}
\href{https://openai.com/blog/whisper/}{OpenAI whisper}
\subsubsection{Real time description}
\subsubsection{Interfaces}
\href{https://tech.fb.com/ar-vr/2021/03/inside-facebook-reality-labs-wrist-based-interaction-for-the-next-computing-platform/}{emg}
\subsubsection{Text to sound}
Complex acoustic environments are possible using \href{https://anonymous.4open.science/w/iclr2023_samples-CB68/report.html}{text to sound} prompting. 
\subsection{Virtual humans}
\subsubsection{Real time human to avatar mapping}
\subsection{AI actors}
\subsubsection{Faces}
\subsubsection{Voices}
\subsubsection{Autonomous tasks}

Extrinsic AI actors which link multiple\\ intrinsic virtual spaces.\\
Bespoke news and current affairs synthesis\\
Bespoke interactive subject matter training\\
bots that bring you what you want as bespoke audio visual packages
\subsection{Governance and safeguarding}
\subsubsection{Governance in the Virtual Reality Space}
The governance of the virtual world will be a critical element in the success of the Metaverse. The virtual world will need to be policed and governed in a way that will not only protect the rights of the citizens of this new digital environment but also protect them from cybercrime. Governments and regulatory bodies will play a key role in the governance of the virtual world, but so will the industry and businesses. Nair et al describe the ``unprecedented privacy risks'' of the metaverse, finding that wearing a headset can currently reveal 25 data points about the user, simply by analysis of the data \cite{nair2022exploring}. This included inference about ethnicity, disability, and economic status. Strong data protection laws will be needed to safeguard privacy, data ownership and reduce the risk of data breaches. The governance of the virtual world will be critical to success, safeguarding will be needed to protect citizens from cyberattacks.
\subsubsection{Safeguarding in the Metaverse}
When it comes to safeguarding in the Metaverse, people need to be made aware of the risk of using VR technology. There are still many questions around the health implications of using VR and the impact it may have on a person’s eyesight. In terms of safeguarding in the Metaverse, this is just one area that needs to be addressed. Users will also need to be made aware of the risks of hacking. Users will need to be educated on the need to be careful when it comes to sharing personal information and be careful what websites they access on a virtual computer. They will need to be made aware of the potential risk of having malware installed on their computer by visiting untrusted websites. Users will also need to be made aware of the potential risk of being manipulated in the virtual world. This risk is particularly high when it comes to children who are growing up in the digital world. They will need to be educated on the potential risks of being groomed or manipulated in the Metaverse.
\subsubsection{How to fight against cybercrime in the Metaverse?}
The best way to fight against cybercrime in the Metaverse is to educate the general public on the potential risks and dangers in order to prevent them from being targeted. This can be done through various channels and mediums, such as social media, blogs and podcasts. People will need to be made aware of the risks of opening emails or clicking on links sent by unknown people. They will also need to be aware of the risks of clicking on ads and links that may lead them to websites that host malware or that steal personal information.

\href{https://www.whitehouse.gov/ostp/ai-bill-of-rights/}{AI bill of rights}\\

Roblox \href{https://www.bbc.co.uk/news/technology-48450604}{in BBC news} for child exploitation.

