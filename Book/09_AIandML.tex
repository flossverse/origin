\section{Augmented intelligence and ML}
\subsection{Novel VP render pipeline}
Putting the ML image generation on the end of a real-time tracked camera render pipeline might remove the need for detail in set building. To describe how this might work, the set designer, DP, director, etc will be able to ideate in a headset based metaverse of the set design, dropping very basic chairs, windows, light sources whatever. There is -no need- then to create a scene in detail. If the interframe consistency (img2img) can deliver then the output on the VP screen can simply inherit the artistic style from the text prompts, and render production quality from the basic building blocks. Everyone in the set (or just DP/director) could then switch in headset to the final output and ideate (verbally) to create the look and feel (lens, bokeh, light, artistic style etc). This isn’t ready yet as the frames need to generate much faster (100x), but it’s very likely coming in months not years. This ``next level pre-vis'' is being trailed by Pathway in the Vircadia collaborative environment described in this book, and can be seen illustrated in Figure \ref{fig:vircadiasd}.\par
\begin{figure}[ht]\centering 	\includegraphics[width=\linewidth]{vircadiasd}
	\caption{Top panel is a screen grab from Vircadia and the bottom panel is a quick pass through img2img from Stable Diffusion.}
	\label{fig:vircadiasd}
\end{figure}

This can be done now through the use of camera robots. A scene can be built in basic outline, the camera tracks can be encoded into the robot, and the scene can be rapidly post rendered by Stability with high inter frame consistency.\par
With the help of AI projects such as \href{https://nv-tlabs.github.io/LION/}{LION} it may be possible to pass simple geometry and instructions to ML systems which can create complex textured geometry back into the scene.
\begin{figure}[ht]\centering 	\includegraphics[width=\linewidth]{robotvp}
	\caption{Robot VP}
	\label{fig:robotvp}
\end{figure}
\subsection{Accessibility}
\subsubsection{Real time transcription}
Real-time language translation can be applied to text interfaces within metaverse applications. This can be useful in situations where users are typing or reading text, rather than speaking.\par
To apply NMT to text interfaces in the metaverse, the algorithm can be integrated into the interface itself. When a user types text in a specific language, the NMT algorithm can automatically detect the language and generate a translation in the desired language. This can be done in real-time, allowing for fast and seamless communication between users speaking different languages. NMT algorithms are well-suited for use in text interfaces, allowing for fast and accurate translations between multiple languages. As the technology continues to advance, we can expect to see more and more applications of NMT in the metaverse.
\subsubsection{Real time translation}
One of its most impressive recent applications is real-time language translation. In this section we will explore how this technology works, and how it can be used in metaverse applications.\par
Real-time language translation refers to the ability of a machine learning model to instantly translate spoken or written text from one language to another. This is different from traditional translation methods, which often involve human translators and can be slow and error-prone.\par
One of the key technologies behind real-time language translation is neural machine translation (NMT). This is a type of machine learning algorithm that is based on neural networks. NMT algorithms are trained on large datasets of text that has been translated by human experts. This allows the algorithm to learn the patterns and nuances of each language, which it can then use to generate accurate translations.\par
One of the key references for the use of neural machine translation in real-time language translation is the paper "Neural Machine Translation by Jointly Learning to Align and Translate" by Bahdanau et al \cite{bahdanau2014neural}. This paper describes the use of a neural network-based approach to machine translation, which has shown impressive results in terms of accuracy and speed.\par
One of the key advantages of NMT is its ability to handle complex and varied sentences. Traditional translation algorithms often rely on fixed rules and dictionaries, which can be limiting. NMT algorithms, on the other hand, can learn to handle a wide range of sentence structures and vocabulary. This makes them well-suited for translating natural languages, which are often full of irregularities and exceptions.\par
Another advantage of NMT is its ability to handle multiple languages at once. Traditional translation algorithms often require the user to specify the source and target languages, but NMT algorithms can automatically detect the languages of the input and output text. This makes them well-suited for use in metaverse applications, where users may be speaking different languages at the same time.\par
One of the challenges of using NMT in metaverse applications is the need for real-time performance. Metaverse applications often involve fast-paced interactions, and any delay in language translation can hinder the user experience. To overcome this challenge, NMT algorithms can be optimized for speed, using techniques such as parallel processing and batching. It seems likely that in our proposed systems we will require API calls to external services for this functionality, and this will almost certainly incur a cost.\par
The use of NMT in metaverse applications is also an active area of research, with a number of papers exploring the potential of this technology. For example, the paper "Real-Time Neural Machine Translation for Virtual Reality" by Chen et al. describes the use of NMT algorithms in virtual reality environments, showing how they can be used to support real-time communication between users speaking different languages.\par
Overall, the use of machine learning for real-time language translation is a rapidly-evolving field, with many exciting developments and applications. As the technology continues to advance, we can expect to see even more impressive results and applications in the future.
\href{https://openai.com/blog/whisper/}{OpenAI whisper}
\subsubsection{Real time description}
\subsubsection{Interfaces}
\href{https://tech.fb.com/ar-vr/2021/03/inside-facebook-reality-labs-wrist-based-interaction-for-the-next-computing-platform/}{emg}
\subsubsection{Text to sound}
Complex acoustic environments are possible using \href{https://anonymous.4open.science/w/iclr2023_samples-CB68/report.html}{text to sound} prompting. 
\subsection{Virtual humans}
\subsubsection{Real time human to avatar mapping}
\subsection{AI actors}
\subsubsection{Faces}
\subsubsection{Voices}
\subsubsection{Autonomous tasks}

Extrinsic AI actors which link multiple\\ intrinsic virtual spaces.\\
Bespoke news and current affairs synthesis\\
Bespoke interactive subject matter training\\
bots that bring you what you want as bespoke audio visual packages
\subsection{Governance and safeguarding}
\subsubsection{Governance in the Virtual Reality Space}
The governance of the virtual world will be a critical element in the success of the Metaverse. The virtual world will need to be policed and governed in a way that will not only protect the rights of the citizens of this new digital environment but also protect them from cybercrime. Governments and regulatory bodies will play a key role in the governance of the virtual world, but so will the industry and businesses. Nair et al describe the ``unprecedented privacy risks'' of the metaverse, finding that wearing a headset can currently reveal 25 data points about the user, simply by analysis of the data \cite{nair2022exploring}. This included inference about ethnicity, disability, and economic status. Strong data protection laws will be needed to safeguard privacy, data ownership and reduce the risk of data breaches. The governance of the virtual world will be critical to success, safeguarding will be needed to protect citizens from cyberattacks.
\subsubsection{Safeguarding in the Metaverse}
When it comes to safeguarding in the Metaverse, people need to be made aware of the risk of using VR technology. There are still many questions around the health implications of using VR and the impact it may have on a person’s eyesight. In terms of safeguarding in the Metaverse, this is just one area that needs to be addressed. Users will also need to be made aware of the risks of hacking. Users will need to be educated on the need to be careful when it comes to sharing personal information and be careful what websites they access on a virtual computer. They will need to be made aware of the potential risk of having malware installed on their computer by visiting untrusted websites. Users will also need to be made aware of the potential risk of being manipulated in the virtual world. This risk is particularly high when it comes to children who are growing up in the digital world. They will need to be educated on the potential risks of being groomed or manipulated in the Metaverse.
\subsubsection{How to fight against cybercrime in the Metaverse?}
The best way to fight against cybercrime in the Metaverse is to educate the general public on the potential risks and dangers in order to prevent them from being targeted. This can be done through various channels and mediums, such as social media, blogs and podcasts. People will need to be made aware of the risks of opening emails or clicking on links sent by unknown people. They will also need to be aware of the risks of clicking on ads and links that may lead them to websites that host malware or that steal personal information.

\href{https://www.whitehouse.gov/ostp/ai-bill-of-rights/}{AI bill of rights}\\

Roblox \href{https://www.bbc.co.uk/news/technology-48450604}{in BBC news} for child exploitation.

